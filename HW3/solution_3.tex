\subsubsection{Part A}
\begin{proof}
    
Pick an arbitrary assignment $x_t$ with distance $D_t > 0$ from a satisfying $x^\star$, and assume $x_t$ does not satisfy $\Phi$ (in which case the algorithm would terminate).

Because $x^\star$ satisfies $\Phi$, every clause $C$ contains at least one literal $\ell^\star \in C$ set to true under the assignment $x^\star$. Conversely, as $C$ is unsatisfied under $x_t$, every literal of $C$ must be false in $x_t$. In particular, we have that $\ell^\star \in C$ is false in $x_t$. Because the algorithm picks a literal $\ell \in C$ u.a.r. among the $k$ possible choices, and at least one of them ($\ell^\star$) would decrease the Hamming distance by $1$, it follows that 
$$\Pr[D_{t+1} = D_t - 1 \,|\, x_t] \ge \frac{1}{k}$$
\end{proof}


\subsubsection{Part B}

\begin{proof}
We assume the satisfying assignment $x^\star$ is unique, as per clarifications on the exercise. 

Consider the event that each of the first $d$ steps decrease the distance by exactly $1$. From part $(a)$, we have that 

$$\Pr\left[ \bigcap_{t=0}^{d-1} \{D_{t+1} = D_t -1\}\right] = \prod_{t=0}^{d-1} \Pr[D_{t+1} = D_t - 1 \,|\, x_t] \ge \left( \frac{1}{k}\right)^d$$

Thus:

$$\Pr[\text{ hit } 0 \text{ within } d \text{ steps} \, | \, D_0 = d \,] \ge \left( \frac{1}{k}\right)^d$$
\end{proof}


\subsubsection{Part C}

\begin{proof}
the result from part $(b)$ gives us

$$\Pr[\text{ success in one try } \, | \, D_0 = d \,] \ge \left( \frac{1}{k}\right)^d$$

From which it immediately follows that 

$$
\begin{align*}
\Pr[ \text{ success in one try }] \ge \mathbb{E}\left[ \left(\frac{1}{k}\right)^{D_0}\right] &= \prod_{i=1}^n \mathbb{E}\left[ \left(\frac{1}{k}\right)^{Z}\right], \quad Z \sim \text{Ber}(1/2) \\
&= \prod_{i=1}^n \left( \frac{1}{2} + \frac{1}{2} \cdot \frac{1}{k}\right) \\
&= \left( \frac{k + 1}{2k}\right)^n
\end{align*}
$$
Where we simply used the MGF of $D_0 \sim \text{Bin}(n, 1/2).$
\end{proof}


\subsubsection{Part D}

The event from $(b)$, namely:
$$D_0 = d \Longrightarrow \Pr[\text{ hit 0 within } d \text{ steps }] \ge (1/k)^d$$
is fully contained in the first $d$ steps. Because we consider the K-SAT problem over $n$ variables, it is clear that the maximal possible initial distance is $D_0 = n$, i.e. when all variables from $x_0 \sim \mathcal{U}\{0,1\}^n$ differs from the satisfying assignment $x^\star$. Thus, $D_0 = d \le n$, and choosing $T=n$ suffices to capture the event fully. 

From part $(c)$, we know that the probability of success in one try is:

$$P \ge \left( \frac{k+1}{2k}\right)^n$$

Therefore the expected number of necessary tries until success is 

$$\mathbb{E}[\text{necessary tries until success}] = O(1/P) = O\left( \left(\frac{2k}{k+1}\right)^n\right)$$

Each try costs $T=n$ time, and thus:

$$\mathbb{E}[\text{time}] = O\left(n \left(\frac{2k}{k+1}\right)^n\right) = \tilde O \left( \left(\frac{2k}{k+1}\right)^n\right)$$



