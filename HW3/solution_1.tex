\begin{proof}

Consider a color set $C$ with cardinality $|C| := Q$. Let us color each vertex of $G = (V,E)$ uniformly at random:

$$\chi(v) \stackrel{\text{i.i.d.}}{\sim} \mathcal{U}\{Q\}, \quad \forall v \in V$$

For each vertex $v \in V$ all colours $c \in C$, and any subset $S \subseteq N(v)$ of size $|S| = \beta + 1$, define the bad event:

$$B(v, c, S) = \{\text{all vertices in S have color } c\}$$

If such a bad event exists, then by definition, the coloring $\chi$ is not $\beta$-frugal over $G$.

From the uniform color assignment, we have that the probability $p$ of the bad event occuring for a fixed tuple $(v,c,S)$ is given by:

$$\mathbb{P}[B(v,c,S)] = Q^{-(\beta + 1)} \stackrel{\triangle}{=} p$$

Let us now find an upper-bound for the number of events $d^\star$ on which $B(u,c,S)$ depends, for a fixed vertex $u \in S$:
\begin{enumerate}
    \item Because each color is assigned independently for each vertex, the event $B(u,c,S)$ only depends on the colors in $S$. In other words, for the events $B(u,c,S)$ and $B(w,\tilde c, \tilde S)$ to be dependent, it must be the case that $u \in \tilde S \subseteq N(w)$, i.e. $w$ and $u$ must be neighbors.
    
    By hypothesis, we know that $\delta(v) \le \Delta$ for all $v \in V$, so there are at most $\Delta$ choices for an initial vertex. 

    \item We have at most $\left( \begin{smallmatrix} \Delta - 1 \\ \beta \end{smallmatrix}\right)$ choices for the remaining $\beta$ vertices in $\tilde S \subseteq N(w) \backslash\{u\}$.

    \item We have $Q$ choices for the color $\tilde c \in C$.
\end{enumerate}
Thus, 

$$d^\star \le \Delta 
\begin{pmatrix}
    \Delta -1 \\
    \beta
\end{pmatrix} 
Q -1$$

Where we subtracted $1$ for the event itself. Union-bounding over the fixed cardinality $|S| = \beta + 1$ yields an upper-bound to the total number of events $d$ that any non-fixed event depends on:

$$
\begin{align*}
d &\le (\beta + 1)d^\star \\
&\le Q(\beta +1)\Delta \begin{pmatrix} \Delta -1 \\ \beta\end{pmatrix} -1 \\
&\le Q (\beta +1) \Delta \left( \frac{e\Delta}{\beta}\right)^\beta -1\\
&= Q(\beta +1)\left(\frac{e}{\beta}\right)^\beta \Delta^{\beta + 1} -1 \stackrel{\triangle}{=} QC_\beta \Delta^{\beta + 1} -1
\end{align*}
$$

Where $C_\beta = (\beta +1)\left(\frac{e}{\beta}\right)^\beta$, and where we used stirling's bound in the third line:

$$k! \ge \left( \frac{k}{e}\right)^k \Longrightarrow \frac{n^k}{k!} \le \frac{n^k}{(k/e)^k} = \left( \frac{en}{k}\right)^k$$

The Local Lovász Lemma states that, if $ep(d+1) \le 1$, then with positive probability, no bad event occurs, meaning a $\beta$-frugal coloring exists. 
Thus, plugging in the computed value for $p$ and the above bound for $d$ into the LLL inequality, it suffices to have

$$
eQ^{-(\beta +1)} \cdot QC_\beta \Delta^{\beta + 1} \le 1 \
\Longrightarrow eC_\beta Q^{-\beta}\Delta^{\beta +1} \le 1 \\
$$

Finally, solving for $Q$ gives:

$$Q^\beta \ge eC_\beta \Delta^{\beta +1} \Longrightarrow Q \ge (eC_\beta)^{1/\beta } \Delta^{1 + 1/\beta}$$

And thus
$$Q =O\left( \Delta^{1 + 1/\beta}\right)$$
as required. 


\end{proof}