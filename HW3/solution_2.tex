We aim to prove that $L$ is a Lipschitz function, and then apply McDiarmid's inequality to obtain the desired concentration bound.

\begin{proof}
We claim that $L$ is a Lipschitz function.
That is, for any two configurations $X=(X_1, \dots, X_n)$ and $X'=(X_1, \dots, X_i', \dots, X_n)$ that differ only in one coordinate $i$, the following holds:
$$
|L(X) - L(X')| \le c
$$
for some constant $c$.

Let $T$ be the minimum spanning tree (MST) of $X$, and let $T'$ be the tree obtained from $T$ by replacing the vertex $X_i$ with $X_i'$.
Although $T'$ may not be the MST of $X'$, it is still a valid spanning tree on the modified set of points.
let $T'^\star$ denote the MST of $X'$, then:
$$
\operatorname{len}(T'^\star) \le \operatorname{len}(T')
$$

Since all points lie within the unit square $[0,1]^2$, the maximum Euclidean distance between any two point is
$$
\sqrt{1^2+1^2}=\sqrt{2}
$$

Given the fact that in planar MST, each vertex has maximum degree of 5.
Therefore, when a single point $X_i$ is moved to $X_i'$, at most five edges are affected, each by changing by at most $\sqrt{2}$ in length.
Hence
$$
|\operatorname{len}(T) - \operatorname{len}(T')| \le 5\sqrt{2}
$$
Since $\operatorname{len}(T'^\star) \le \operatorname{len}(T')$, it follows that:
$$
|L(X) - L(X')| \le |\operatorname{len}(T) - \operatorname{len}(T'^\star)| \le 5\sqrt{2}
$$
Thus, $L$ is $c$-Lipschitz with $c=5\sqrt{2}$.

We can now apply McDiarmid's inequality, which states that if $L$ is $c_i$-Lipschitz in each coordinate, then:
$$
\Pr\big[ |L - \mu| \geq t \big] \leq 2\exp\!\left( \frac{-2t^2}{\sum_{i=1}^n c_i^2} \right)
$$
Since each $c_i=5\sqrt{2}$, we have
$$
\sum_{i=1}^n c_i^2 = n\cdot(5\sqrt{2})^2 = 50n
$$
Setting $t=\varepsilon n$, we obtain:
$$
\Pr\big[ |L - \mu| \geq \varepsilon n \big] \leq 2\exp\!\left( \frac{-2\varepsilon^2n^2}{50n} \right) = 2\exp\!\left( \frac{-\varepsilon^2n}{25} \right)
$$

\end{proof}