 \documentclass[11pt,a4paper,twoside]{article}

\newcommand{\assignment}{Homework 1 Solutions}
\newcommand{\studentone}{Barras Simon <simon.barras@epfl.ch>}
\newcommand{\studenttwo}{Doll Zachary <zachary.doll@epfl.ch>} % Comment out if there is no second student
%\newcommand{\studentthree}{Student 3 name + email} % Comment out if there is no third student


\usepackage[T1]{fontenc}
\usepackage[utf8]{inputenc}

\usepackage{mathpazo}

\usepackage{amsmath,amssymb,amsfonts}

\usepackage{nicefrac}

\newcommand{\nf}{\nicefrac}

\usepackage[backref,colorlinks,citecolor=blue,bookmarks=true]{hyperref}

\usepackage[shortlabels]{enumitem}
\newlist{alphenum}{enumerate}{2}
\setlist[alphenum]{label=(\alph*),beginpenalty=10000}

\newlist{menum}{enumerate}{2}
\setlist[menum]{label=(\arabic*),beginpenalty=10000}

\usepackage[hmargin=2.8cm,top=2.8cm,bottom=3.5cm,nohead,footskip=48pt]{geometry}

\usepackage{amsthm}
\newtheorem{theorem}{Theorem}
\newtheorem{lemma}{Lemma}
\newtheorem{property}{Property}
\newtheorem{fact}{Fact}
\newtheorem{claim}{Claim}

\theoremstyle{definition}
\newtheorem{definition}{Definition}
\newtheorem{example}{Example}

\usepackage{bm}

\parindent0pt
\parskip1ex

% reset mathcal fonts to detault
\DeclareMathAlphabet{\mathcal}{OMS}{cmsy}{m}{n}

\makeatletter
\newcommand{\header}{%
\mbox{}
\vspace*{-1.8cm}
\par
\parbox[b]{13.1cm}{\raggedright
    {\large\bfseries Topics in Theoretical Computer Science: Randomized Algorithms}\\[.5ex]    
    EPFL, Fall semester, 2025 \\[2ex]
}%
%\parbox[b]{4.3cm}{\hspace*{0pt}\fbox{\includegraphics[width=4.3cm]{RSP-LEO-logo.pdf}}}%

  
\vspace*{-.3cm}

\rule{\textwidth}{.5pt}
\vspace*{-.2cm}

{\Large\bfseries \assignment } \\[2ex]
{\large \studentone \\[1ex]
\@ifundefined{studenttwo}{}{
    \studenttwo \\[1ex]
}
\@ifundefined{studentthree}{}{
    \studentthree \\[1ex]
}
}
\vspace*{-0.5cm}

\rule{\textwidth}{.5pt}

\vspace*{-.3cm}

}
\makeatother


\begin{document}

\header

% SORTING ALMOST CORRECT SORTING
% SUBMODULAR PORTFOLIO OPTIMIZATION
% WHY STRAIGHTFORWARD USE OF PREDICTIONS IS DANGEROUS (SCHEDULING OR CACHING)
% MECHANISM DESIGN ON THE LINE?



\section{Problem 1 (Lonely Vertices)}
Consider the Erd\H{o}s-R\'enyi random graph
$G(n,p)$ and suppose $\epsilon > 0$ is some constant. Show that:
\begin{alphenum}
    \item if $p \geq \frac{(1+\epsilon)\ln n}{n-1}$ then the graph has at no
    isolated vertices with probability $1-o(1)$.
    \item if $p \leq \frac{(1-\epsilon)\ln n}{n-1}$ then the graph has at
    least one isolated vertex with probability $1-o(1)$.
\end{alphenum}

\subsection{Solution}
\begin{proof}

Consider a color set $C$ with cardinality $|C| := Q$. Let us color each vertex of $G = (V,E)$ uniformly at random:

$$\chi(v) \stackrel{\text{i.i.d.}}{\sim} \mathcal{U}\{Q\}, \quad \forall v \in V$$

For each vertex $v \in V$ all colours $c \in C$, and any subset $S \subseteq N(v)$ of size $|S| = \beta + 1$, define the bad event:

$$B(v, c, S) = \{\text{all vertices in S have color } c\}$$

If such a bad event exists, then by definition, the coloring $\chi$ is not $\beta$-frugal over $G$.

From the uniform color assignment, we have that the probability $p$ of the bad event occuring for a fixed tuple $(v,c,S)$ is given by:

$$\mathbb{P}[B(v,c,S)] = Q^{-(\beta + 1)} \stackrel{\triangle}{=} p$$

Let us now find an upper-bound for the number of events $d^\star$ on which $B(u,c,S)$ depends, for a fixed vertex $u \in S$:
\begin{enumerate}
    \item Because each color is assigned independently for each vertex, the event $B(u,c,S)$ only depends on the colors in $S$. In other words, for the events $B(u,c,S)$ and $B(w,\tilde c, \tilde S)$ to be dependent, it must be the case that $u \in \tilde S \subseteq N(w)$, i.e. $w$ and $u$ must be neighbors.
    
    By hypothesis, we know that $\delta(v) \le \Delta$ for all $v \in V$, so there are at most $\Delta$ choices for an initial vertex. 

    \item We have at most $\left( \begin{smallmatrix} \Delta - 1 \\ \beta \end{smallmatrix}\right)$ choices for the remaining $\beta$ vertices in $\tilde S \subseteq N(w) \backslash\{u\}$.

    \item We have $Q$ choices for the color $\tilde c \in C$.
\end{enumerate}
Thus, 

$$d^\star \le \Delta 
\begin{pmatrix}
    \Delta -1 \\
    \beta
\end{pmatrix} 
Q -1$$

Where we subtracted $1$ for the event itself. Union-bounding over the fixed cardinality $|S| = \beta + 1$ yields an upper-bound to the total number of events $d$ that any non-fixed event depends on:

$$
\begin{align*}
d &\le (\beta + 1)d^\star \\
&\le Q(\beta +1)\Delta \begin{pmatrix} \Delta -1 \\ \beta\end{pmatrix} -1 \\
&\le Q (\beta +1) \Delta \left( \frac{e\Delta}{\beta}\right)^\beta -1\\
&= Q(\beta +1)\left(\frac{e}{\beta}\right)^\beta \Delta^{\beta + 1} -1 \stackrel{\triangle}{=} QC_\beta \Delta^{\beta + 1} -1
\end{align*}
$$

Where $C_\beta = (\beta +1)\left(\frac{e}{\beta}\right)^\beta$, and where we used stirling's bound in the third line:

$$k! \ge \left( \frac{k}{e}\right)^k \Longrightarrow \frac{n^k}{k!} \le \frac{n^k}{(k/e)^k} = \left( \frac{en}{k}\right)^k$$

The Local Lovász Lemma states that, if $ep(d+1) \le 1$, then with positive probability, no bad event occurs, meaning a $\beta$-frugal coloring exists. 
Thus, plugging in the computed value for $p$ and the above bound for $d$ into the LLL inequality, it suffices to have

$$
eQ^{-(\beta +1)} \cdot QC_\beta \Delta^{\beta + 1} \le 1 \
\Longrightarrow eC_\beta Q^{-\beta}\Delta^{\beta +1} \le 1 \\
$$

Finally, solving for $Q$ gives:

$$Q^\beta \ge eC_\beta \Delta^{\beta +1} \Longrightarrow Q \ge (eC_\beta)^{1/\beta } \Delta^{1 + 1/\beta}$$

And thus
$$Q =O\left( \Delta^{1 + 1/\beta}\right)$$
as required. 


\end{proof}

\section{Problem 2 (Cutting it Fine)}
Given a connected undirected (and unweighted,
for simplicity) $n$-vertex graph $G$, a $k$-cut is a paritition of
the vertex set into $k$ non-empty parts $S_1, S_2, \ldots, S_k$. The
size of the $k$-cut is the number of edges that cross between
distinct parts in this partition. Consider the following algorithm:
\begin{itemize}
    \item (Phase 1) As long as the number of vertices in $G$ is more
          than $2k-2$, pick a random edge and contract it.
    \item (Phase 2) Now we have a graph on $2k-2$ vertices. For each of
          these vertices, choose a random label from
          $\{L_1,L_2,\ldots, L_k\}$, contract vertices with the same label,
          and output the resulting cut.
\end{itemize}
Show the following:
\begin{alphenum}
    \item If the min $k$-cut size is $\lambda$, then
    $\lambda \leq 2(k-1)\frac{m}{n}$.
    \item Any fixed min $k$-cut survives the first phase with
    probability at least $1/{n \choose {2(k-1)}}$.
    \item Conditioned on surviving the first phase, it is output in the
    second phase with probability at least
    $\frac{k!}{k^k}\cdot \frac{1}{k^{k-2}}$.
\end{alphenum}
Hence, this gives an $\approx n^{2(k-1)}$-time algorithm for the
$k$-cut problem.



\subsection{Solution}
We aim to prove that $L$ is a Lipschitz function, and then apply McDiarmid's inequality to obtain the desired concentration bound.

\begin{proof}
We claim that $L$ is a Lipschitz function.
That is, for any two configurations $X=(X_1, \dots, X_n)$ and $X'=(X_1, \dots, X_i', \dots, X_n)$ that differ only in one coordinate $i$, the following holds:
$$
|L(X) - L(X')| \le c
$$
for some constant $c$.

Let $T$ be the minimum spanning tree (MST) of $X$, and let $T'$ be the tree obtained from $T$ by replacing the vertex $X_i$ with $X_i'$.
Although $T'$ may not be the MST of $X'$, it is still a valid spanning tree on the modified set of points.
let $T'^\star$ denote the MST of $X'$, then:
$$
\operatorname{len}(T'^\star) \le \operatorname{len}(T')
$$

Since all points lie within the unit square $[0,1]^2$, the maximum Euclidean distance between any two point is
$$
\sqrt{1^2+1^2}=\sqrt{2}
$$

Given the fact that in planar MST, each vertex has maximum degree of 5.
Therefore, when a single point $X_i$ is moved to $X_i'$, at most five edges are affected, each by changing by at most $\sqrt{2}$ in length.
Hence
$$
|\operatorname{len}(T) - \operatorname{len}(T')| \le 5\sqrt{2}
$$
Since $\operatorname{len}(T'^\star) \le \operatorname{len}(T')$, it follows that:
$$
|L(X) - L(X')| \le |\operatorname{len}(T) - \operatorname{len}(T'^\star)| \le 5\sqrt{2}
$$
Thus, $L$ is $c$-Lipschitz with $c=5\sqrt{2}$.

We can now apply McDiarmid's inequality, which states that if $L$ is $c_i$-Lipschitz in each coordinate, then:
$$
\Pr\big[ |L - \mu| \geq t \big] \leq 2\exp\!\left( \frac{-2t^2}{\sum_{i=1}^n c_i^2} \right)
$$
Since each $c_i=5\sqrt{2}$, we have
$$
\sum_{i=1}^n c_i^2 = n\cdot(5\sqrt{2})^2 = 50n
$$
Setting $t=\varepsilon n$, we obtain:
$$
\Pr\big[ |L - \mu| \geq \varepsilon n \big] \leq 2\exp\!\left( \frac{-2\varepsilon^2n^2}{50n} \right) = 2\exp\!\left( \frac{-\varepsilon^2n}{25} \right)
$$

\end{proof}

\section{Problem 3 (Cyclic Changes)}
An $\ell$-cycle in a graph is a cycle with \emph{at
    most} $\ell$ nodes. In this problem, we want to show there exist
graphs with many edges, and no \emph{short} cycles. It is easy to
construct such graphs with $\Omega(n)$ edges---in fact, a tree has
$n-1$ edges and no cycles at all!  We want slightly denser graphs
with no $\ell$-cycles for any constant $\ell$.
\begin{alphenum}
    \item Consider the graph $G(n, p)$ for some $p \in [0,1]$. Calculate
    the probability that some sequence of $\ell$ vertices is a cycle,
    and hence calculate the expected number of $\ell$-cycles.

    \item (Do not submit.) Note that setting $p \approx 1/n$ means the
    expected number of $\ell$-cycles is $o(1)$, but the expected
    number of edges is $O(n)$---which is not very interesting!

    \item Now consider the following two part algorithm: (i) first pick
    $G \sim G(n,p)$, and then (ii) for each $\ell$-cycle in $G$,
    delete an arbitrary edge on it. By construction this graph has no
    $\ell$-cycles. Show that setting $p = n^{\frac{2 - \ell}{\ell-1}}$
    ensures that the expected number of edges in the resulting graph
    is $m := \Theta(n^{1 + \frac{1}{\ell-1}})$. Hence, infer that there exist
    graphs with $m = \omega(n)$ edges and no $\ell$-cycles.
\end{alphenum}




\subsection{Solution}
\subsubsection{Part A}
\begin{proof}
    
Pick an arbitrary assignment $x_t$ with distance $D_t > 0$ from a satisfying $x^\star$, and assume $x_t$ does not satisfy $\Phi$ (in which case the algorithm would terminate).

Because $x^\star$ satisfies $\Phi$, every clause $C$ contains at least one literal $\ell^\star \in C$ set to true under the assignment $x^\star$. Conversely, as $C$ is unsatisfied under $x_t$, every literal of $C$ must be false in $x_t$. In particular, we have that $\ell^\star \in C$ is false in $x_t$. Because the algorithm picks a literal $\ell \in C$ u.a.r. among the $k$ possible choices, and at least one of them ($\ell^\star$) would decrease the Hamming distance by $1$, it follows that 
$$\Pr[D_{t+1} = D_t - 1 \,|\, x_t] \ge \frac{1}{k}$$
\end{proof}


\subsubsection{Part B}

\begin{proof}
We assume the satisfying assignment $x^\star$ is unique, as per clarifications on the exercise. 

Consider the event that each of the first $d$ steps decrease the distance by exactly $1$. From part $(a)$, we have that 

$$\Pr\left[ \bigcap_{t=0}^{d-1} \{D_{t+1} = D_t -1\}\right] = \prod_{t=0}^{d-1} \Pr[D_{t+1} = D_t - 1 \,|\, x_t] \ge \left( \frac{1}{k}\right)^d$$

Thus:

$$\Pr[\text{ hit } 0 \text{ within } d \text{ steps} \, | \, D_0 = d \,] \ge \left( \frac{1}{k}\right)^d$$
\end{proof}


\subsubsection{Part C}

\begin{proof}
the result from part $(b)$ gives us

$$\Pr[\text{ success in one try } \, | \, D_0 = d \,] \ge \left( \frac{1}{k}\right)^d$$

From which it immediately follows that 

$$
\begin{align*}
\Pr[ \text{ success in one try }] \ge \mathbb{E}\left[ \left(\frac{1}{k}\right)^{D_0}\right] &= \prod_{i=1}^n \mathbb{E}\left[ \left(\frac{1}{k}\right)^{Z}\right], \quad Z \sim \text{Ber}(1/2) \\
&= \prod_{i=1}^n \left( \frac{1}{2} + \frac{1}{2} \cdot \frac{1}{k}\right) \\
&= \left( \frac{k + 1}{2k}\right)^n
\end{align*}
$$
Where we simply used the MGF of $D_0 \sim \text{Bin}(n, 1/2).$
\end{proof}


\subsubsection{Part D}

The event from $(b)$, namely:
$$D_0 = d \Longrightarrow \Pr[\text{ hit 0 within } d \text{ steps }] \ge (1/k)^d$$
is fully contained in the first $d$ steps. Because we consider the K-SAT problem over $n$ variables, it is clear that the maximal possible initial distance is $D_0 = n$, i.e. when all variables from $x_0 \sim \mathcal{U}\{0,1\}^n$ differs from the satisfying assignment $x^\star$. Thus, $D_0 = d \le n$, and choosing $T=n$ suffices to capture the event fully. 

From part $(c)$, we know that the probability of success in one try is:

$$P \ge \left( \frac{k+1}{2k}\right)^n$$

Therefore the expected number of necessary tries until success is 

$$\mathbb{E}[\text{necessary tries until success}] = O(1/P) = O\left( \left(\frac{2k}{k+1}\right)^n\right)$$

Each try costs $T=n$ time, and thus:

$$\mathbb{E}[\text{time}] = O\left(n \left(\frac{2k}{k+1}\right)^n\right) = \tilde O \left( \left(\frac{2k}{k+1}\right)^n\right)$$





\section{Problem 4 (Random 3-SAT)}
 (Taken from Johan H\aa stad's course ``Theoreticians toolkit''
 at KTH.) Constructing a random 3-SAT formula with n variables and
$m = \lceil dn \rceil$ (remember that $\lceil x \rceil$ is the
smallest integer larger than x) clauses is done in the following
way. Randomly take three different variables (all triples being
equally likely). With uniform probability choose one of the eight
ways to negate these variables and make them into a clause. Repeat
with independent randomness until you have $m$ clauses.

\begin{alphenum}
    \item For what value of $d$ is the expected number of satisfying
    assignments $\Theta(1)$?  Call this value $d_0$.

    \item Prove that the formula is likely (probability $1-o(1)$) to be
    unsatisfiable for any constant $d$ such that $d > d_0$.
    \item \textbf{(Extra Credit.)} Prove that the formula remains at least somewhat likely to be
    unsatisfiable also in the case when $d$ is slightly smaller than
    $d_0$. The difficulty of this problem is very much dependent on
    what we mean by ``somewhat likely'' and ``slightly smaller''. The
    exact formulation to prove to get a full score on this problem is
    that there is some constant $d_1 < d_0$ such that for $d = d_1$
    the probability that the corresponding random formula is
    satisfiable is at most $1/2$. The size of $d_0 - d_1$ does not
    matter for your score on the problem and the main property of a
    solution to aim for is a mathematically correct argument.
    \emph{Hint:} A satisfiable formula that does not depend on all its
    variables has many satisfying assignments.

    % As a curiosity we may note that the value of $d$ such that such
    % a formula has probability around $1/2$ of being satisfiable is
    % not known but conjectured to be around $4.2$.
\end{alphenum}


\subsection{solution}
\subsubsection{Part A}
We aim to show that the brute-force algorithm for finding the longest path of length at most $k$ in a graph $G=(V,E)$ runs in $O(n\Delta^k)$, where $n=|V|$ is the number of vertices and $\Delta$ is the maximum degree of any vertex in $G$.

\begin{proof}
The high-level algorithm (Algorithm~\ref{code:longestPathGraph}) enumerates all vertices as potential starting points for the longest path.
For each vertex $v\in V$, it calls a recursive procedure that explores all possible simple paths starting from $v$ with length at most $k$. The outer loop thus contributes as factor of $O(n)$.

\begin{algorithm}
\caption{Find longest path in the graph}\label{code:longestPathGraph}
\begin{algorithmic}
\Procedure{Longest-Path}{$G=(V, E), k$}
\State $P \gets \emptyset$
\For{$ v \in V$}
    \State $Q \gets \operatorname{Longest-Path-Vertex}(G, v, \emptyset, k)$
    \State $P \gets \operatorname{max}(P, Q)$
\EndFor
\State \Return $P$
\EndProcedure
\end{algorithmic}
\end{algorithm}

Algorithm~\ref{code:longestPathVertex} describes the recursive procedure that computes the longest path starting from a fixed vertex $v$.
At each recursive step, the algorithm considers all neighbors $u$ of $v$ (excluding the predecessor vertex $p$) and recursively explores all paths of remaining length $k-1$ starting from $u$.

Since each vertex has at most $\Delta$ neighbors, and the recursion proceeds to depth $k$, the total number of recursive calls can be upper-bounded by $O(\Delta^k)$. This corresponds to exploring all possible paths of length up to $k$ starting from a given vertex.

\begin{algorithm}
\caption{Find longest path of a vertex}\label{code:longestPathVertex}
\begin{algorithmic}
\Procedure{Longest-Path-Vertex}{$G=(V, E), v, p, k$}
\State $P \gets \emptyset$
\If{$k=0$} \State \Return $P$
\EndIf
\For{$ u \in V \text{ where }\{u, v\} \in E \text{ and } u \ne p$}
    \State $Q \gets \operatorname{Longest-Path-Vertex}(G, u, v, k-1)$
    \State $P \gets \operatorname{max}(P, Q)$
\EndFor
\State \Return $P$
\EndProcedure
\end{algorithmic}
\end{algorithm}

Combining both procedures, the overall time complexity is obtained by multiplying the outer $O(n)$ loop with the recursive exploration cost $O(\Delta^k)$.
Therefore, the total running time of the algorithm is
$$
O(n\Delta^k)
$$
\end{proof}





\subsubsection{Part B}
We aim to show that if the graph $G=(V,E)$ is directed and acyclic, there exist an algorithm to find the longest path in time $O(V+E)$.

\begin{proof}
Finding the the longest path in a DAG (Directed Acyclic Graph) can achieved efficiently using a DFS (Depth-First Search) combined with dynamic programming.
This approach ensures that each vertex is processed only once, avoiding redundant computations and achieving the desired linear-time complexity.

The high-level procedure (Algorithm~\ref{code:longestPathDAG}) iterates over all vertices, treating each as potential starting points for the longest path.
For every vertex $v\in V$, it invokes a recursive routines that computes and memorizes the longest path starting from $v$.
The outer loop thus contributes as factor of $O(V)$.

\begin{algorithm}
\caption{Find longest path in a DAG}\label{code:longestPathDAG}
\begin{algorithmic}
\Procedure{Longest-Path-DAG}{$G=(V, E), k$}
\State $\mathcal{P} \gets \{\emptyset\}$ for $|V|$
\For{$ v \in V$}
    \State $\operatorname{Longest-Path-Vertex-DAG}(G, \mathcal{P}, v)$
\EndFor
\State \Return $\operatorname{max}(P \in \mathcal{P})$
\EndProcedure
\end{algorithmic}
\end{algorithm}

Algorithm~\ref{code:longestPathVertexDAG} defines the recursive routine improved with dynamic programming that computes the longest path starting from a fixed vertex $v$.
If the longest path for $v$ has already been computed (i.e., $\mathcal{P}[v] \neq \emptyset$), the result is returned immediately.
Otherwise, for each outgoing edge $(v, u) \in E$, the algorithm recursively computes the longest path starting from $u$ and updates $\mathcal{P}[v]$ accordingly.

\begin{algorithm}
\caption{Find longest path of a vertex in DAG}\label{code:longestPathVertexDAG}
\begin{algorithmic}
\Procedure{Longest-Path-Vertex-DAG}{$G=(V, E), \mathcal{P}, v$}
\If{$\mathcal{P}[v] \ne \emptyset$}
    \Return
\EndIf

\For{$ e=\{v, u\} \in E$}
    \State $\operatorname{Longest-Path-Vertex-DAG}(G, \mathcal{P}, u)$
    \State $\mathcal{P}[v] \gets \operatorname{max}(\mathcal{P}[v], \mathcal{P}[u]+e)$
\EndFor
\EndProcedure
\end{algorithmic}
\end{algorithm}

Each vertex $v$ is visited only once, and during its processing, the algorithm inspects all its outgoing edges.
The total amount of work fo all vertices is the total number of edges $E$.

Thus combining bot procedures yields an algorithm that computes the longest path in a DAG in linear time.
$$
O(V+E)
$$
\end{proof}





\subsubsection{Part C}
We claim that if we assign a random ordering to all vertices of an undirected graph $G=(V,E)$ and orient each edge from the vertex with smaller index to the one with larger index, we obtain a directed acyclic graph (DAG) $\vec G=(V,E)$.

Now, consider construction $n$ independent random orientations $\vec G_1,\dots,\vec G_n$.
We want to show that, for
$$
k=c\frac{\log n}{\log\log n}
$$
with a sufficiently large constant $c>0$, the probability that at least one of these DAGs contains a path of length $k$ is at least $\tfrac{1}{2}$:
$$
\Pr\left[\exists i,  \space P\subseteq \vec G_i, \space\operatorname{len}(P)=k\right] \ge \frac{1}{2}
$$

\begin{proof}
Fix a specific path $P$ of length $k$ in $G$.
It consists of $k+1$ vertices.
When we randomly permute the vertices, there are $(k+1)!$ possible relative orderings of these vertices, and only one of them produces an increasing order consistent with the directed edges of $\vec G$.
Hence
$$
\Pr\left[P\subseteq\vec G\right] = \frac{1}{(k+1)!} =: p
$$
The longer the path, the smaller this probability becomes.
Let $P^\star$ be a longest path in $G$.
For our family of random orientations, the probability that at least one of the $\vec G_i$ contains $P^\star$ is
$$
\Pr\left[P\in \bigcup_{i=1}^n\vec G_i\right] \ge \Pr\left[P^\star\in \bigcup_{i=1}^n\vec G_i \right] = 1 - \Pr\left[P^\star\notin \bigcup_{i=1}^n\vec G_i \right] = 1 -(1-p)^n
$$
Using the standard exponential bound $(1-p)^n \le e^{-pn}$.
$$
\Pr\left[P^\star\in \bigcup_{i=1}^n\vec G_i \right] \ge 1-e^{-pn} = 1-\exp\left(-n\frac{1}{(k+1)!}\right) = 1-\exp\left(-n\frac{1}{\left(c\frac{\log n}{\log\log n}+1\right)!}\right)
$$

Using Stirling's bound
$$
(k+1)! \ge \sqrt{2\pi(k+1)}\left(\frac{k+1}{e}\right)^{k+1}
$$
we obtain
$$
\frac{1}{(k+1)!} \le \frac{1}{\sqrt{2\pi(k+1)}}\left(\frac{e}{k+1}\right)^{k+1}
$$
Substituting $k = c\frac{\log n}{\log\log n}$
$$
p \le \frac{1}{\sqrt{2\pi k}}\left(\frac{e\log\log n}{c \log n}\right)^{c\frac{\log n}{\log\log n}}
$$
Hence
$$
pn \le n^{1-c+o(1)}
$$

If we choose any $c>1$, then $pn \to 0$ as $n\to\infty$, implying $1-e^{-pn} \to 0$.
Conversely, if $c<1$, then $pn \to \infty$, and 
$$
1-e^{-pn} \to 1
$$
Thus, there exist a threshold constant $c^\star\approx 1$ such tact for $c<c^\star$ the probability exceeds $\tfrac{1}{2}$ for large $n$.
\end{proof}





\subsubsection{Part D}
Consider a graph $G=(V, E)$ where each vertex is assigned one of $k$ colors, such that no two adjacent vertices share the same color.
A path $P$ is said to be polychromatic if it contains at most one vertex of each color.
Clearly, the length of any such path satisfies $\operatorname{len}(P) \le k$.

We claim that there exist a algorithms running in time $\operatorname{poly}(n,k)2^k$ that finds the longest polychromatic path.

\begin{proof}
Consider a dynamic programming (DP) table defined as:
$$
\operatorname{DP}[C][v] = \begin{cases}
    \operatorname{True} \text{ if there exists a path ending at vertex $v$ with $C$} \\
    \operatorname{False} \text{ otherwise}
\end{cases}
$$
Where $C$ is a polychromatic subset.

The DP is initialized for single-vertex paths:
$$
\operatorname{DP}[\operatorname{col}(v)][v] = \operatorname{True}, \quad\forall v \in V
$$
Then, it is updated recursively:
$$
\operatorname{DP}[C][v] = \bigvee_{u\in\operatorname{N}(v)} \operatorname{DP}[C\setminus\operatorname{col}(v)][u]
$$
Where $N(v)$ denotes the set of neighbors of $v$.

Each entry in the DP table represents whether a polychromatic path with color subset $C$ ends at vertex $v$.

Since there are $2^k$ subsets of colors and $n$ vertices, the table contains $O(n2^k)$ entries.
For each vertex, we may check all its incident edges, leading to a total time complexity of:
$$
O\left(2^k(n+m)\right) = \operatorname{poly}(n, k)2^k
$$
\end{proof}



By repeating the random coloring of the graph $G=(V,E)$ independently $n$ times to generate graphs $G^\bullet_1,\dots,G^\bullet_n$, we claim that for
$$
k = c\log n
$$
with constant $c>0$, the algorithm finds a polychromatic path of length $k$ (if exists) with probability at least $\frac{1}{2}$:
$$
\Pr\left[\exists i,  \space P\subseteq G_i^\bullet, \space\operatorname{len}(P)=k\right] \ge \frac{1}{2}
$$

\begin{proof}
Fix a specific path $P$ of length $k$ in $G$.
The path contains $k$ distinct vertices, each independently assigned one of $k$ color.

There are $k^k$ possible colorings of the vertices along $P$, but only those in which all vertices have distinct colors yield a polychromatic path. The number of distinct-color colorings equals $k!$.
Thus, for a single random coloring:
$$
\Pr\left[\operatorname{p}(P)\in G^\bullet\right] = \frac{k!}{k^k} =: p
$$
Where $\operatorname{p}(P)$ is the polychromatic possible path of $P$.

For our family of random coloring, the probability that at least one of the $G_i^\bullet$ contains a valid $\operatorname{p}(P)$ of length $k$ is
$$
\Pr\left[\exists i:  \space \operatorname{p}(P)\in G_i^\bullet\right] = 1-(1-p)^n \ge 1-e^{-pn} = 1-\exp\left(-n\frac{k!}{k^k}\right)
$$
To analyze this expression asymptotically, we apply Stirling's approximation:
$$
k! \approx\sqrt{2\pi k}\left(\frac{k}{e}\right)^k
$$
Thus,
$$
\frac{k!}{k^k} \approx e^{-k}\sqrt{2\pi k}
$$
Plugging this back, we get:
$$
\Pr\left[\exists i:  \space \operatorname{p}(P)\in G_i^\bullet\right] \ge 1-\exp\left(e^{-k}\sqrt{2\pi k}\right)
$$
Setting $k=c\log n$
$$
ne^{-k}\sqrt{2\pi k} = ne^{-c\log n}\sqrt{2\pi c\log n} = n^{1-c}\sqrt{2\pi c\log n}
$$
For this probability to be at least $\tfrac{1}{2}$, we require the exponent to be at least $\ln 2$. That is, $n^{1-c}\sqrt{\log n} = \Omega(1)$, which holds for any constant $c<1$.

Therefore, for any constant $c<1$,
$$
k = c \log n \Rightarrow \Pr\left[\exists i:  \space \operatorname{p}(P)\in G_i^\bullet\right] \ge \frac{1}{2}
$$
\end{proof}


\end{document}
