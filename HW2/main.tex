\documentclass[11pt,a4paper,twoside]{article}

\newcommand{\assignment}{Homework 2 Solutions}
\newcommand{\studentone}{Barras Simon <simon.barras@epfl.ch>}
\newcommand{\studenttwo}{Doll Zachary <zachary.doll@epfl.ch>} % Comment out if there is no second student
%\newcommand{\studentthree}{Student 3 name + email} % Comment out if there is no third student



\usepackage[T1]{fontenc}
\usepackage[utf8]{inputenc}

\usepackage{mathpazo}

\usepackage{amsmath,amssymb,amsfonts}

\usepackage{nicefrac}

\newcommand{\nf}{\nicefrac}

\usepackage[backref,colorlinks,citecolor=blue,bookmarks=true]{hyperref}

\usepackage[shortlabels]{enumitem}
\newlist{alphenum}{enumerate}{2}
\setlist[alphenum]{label=(\alph*),beginpenalty=10000}

\newlist{menum}{enumerate}{2}
\setlist[menum]{label=(\arabic*),beginpenalty=10000}

\usepackage[hmargin=2.8cm,top=2.8cm,bottom=3.5cm,nohead,footskip=48pt]{geometry}

\usepackage{amsthm}
\newtheorem{theorem}{Theorem}
\newtheorem{lemma}{Lemma}
\newtheorem{property}{Property}
\newtheorem{fact}{Fact}
\newtheorem{claim}{Claim}

\theoremstyle{definition}
\newtheorem{definition}{Definition}
\newtheorem{example}{Example}


\usepackage{bm}

\newcommand{\calE}{\mathcal{E}}
\newcommand{\calG}{\mathcal{G}}
\newcommand{\R}{\mathbb{R}}

\parindent0pt
\parskip1ex

% reset mathcal fonts to detault
\DeclareMathAlphabet{\mathcal}{OMS}{cmsy}{m}{n}

\makeatletter
\newcommand{\header}{%
\mbox{}
\vspace*{-1.8cm}
\par
\parbox[b]{13.1cm}{\raggedright
    {\large\bfseries Topics in Theoretical Computer Science: Randomized Algorithms}\\[.5ex]    
    EPFL, Fall semester, 2025 \\[2ex]
}%
%\parbox[b]{4.3cm}{\hspace*{0pt}\fbox{\includegraphics[width=4.3cm]{RSP-LEO-logo.pdf}}}%

  
\vspace*{-.3cm}

\rule{\textwidth}{.5pt}
\vspace*{-.2cm}

{\Large\bfseries \assignment } \\[2ex]
{\large \studentone \\[1ex]
\@ifundefined{studenttwo}{}{
    \studenttwo \\[1ex]
}
\@ifundefined{studentthree}{}{
    \studentthree \\[1ex]
}
}
\vspace*{-0.5cm}

\rule{\textwidth}{.5pt}

\vspace*{-.3cm}

}
\makeatother


\begin{document}

\header

% SORTING ALMOST CORRECT SORTING
% SUBMODULAR PORTFOLIO OPTIMIZATION
% WHY STRAIGHTFORWARD USE OF PREDICTIONS IS DANGEROUS (SCHEDULING OR CACHING)
% MECHANISM DESIGN ON THE LINE?



\section{Problem 1 (Only Connect!)}

Given an undirected (unweighted) graph
$G = (V,E)$, let $G(p)$ be the random graph where we retain each
edge of $G$ independently with probability $p$.  In lecture \#4, we
saw that setting $p \geq c \frac{\log n}{\lambda}$, where $\lambda$
is the min-cut value in $G$, the graph $G(p)$ is a cut-approximator
for $G$ with probability $1 - o(1)$. In particular, we get the
simpler fact: if $G$ is connected, then $G_p$ is also connected
whp. Let's prove this simpler fact in a different way that does not
use the cut-counting lemma. Consider the following process:

\begin{quote}
    Initialize $G_0 = G$, and define $L = 100 \log n$. For each
    $i = 1, 2, \ldots, L$, let $S_i$ be a set where we pick each edge
    in $G_{i-1}$ independently with probability $1/\lambda$. Contract
    all the edges from $S_i$ in the graph $G_{i-1}$ (and remove
    self-loops) to get $G_{i}$.
\end{quote}

Analyze it as follows:

\begin{alphenum}

    \item For any vertex $v$ in $G_{i-1}$, let $\calG_{v,i}$ be the
    event that the set $S_i$ contains at least one edge incident to
    $v$. Show that $\Pr[\calG_{v,i}] \geq 1 - \nf1e$. Btw, are
    $G_{v,i}$ and $G_{u,i}$ independent?

    \item Let $N_i$ be the number of vertices in $G_i$, so that
    $N_0 = n$. Define the event $\calE_i$ if
    $N_i \leq N_{i-1}\cdot \nf34$. Show that $\Pr[\calE_i] \geq c$,
    for some absolute constant $c > 0$.

    \item Use a Chernoff bound to show that $|N_L| = 1$ with probability
    at least $1-1/\text{poly}(n)$. Please clearly state what random
    variables are you summing over, and why they are independent and
    bounded.

    \item Finally, define $S = \cup_{i = 1}^L S_i$, and note that each
    edge in $G$ belongs to $S$ with probability at most $L/\lambda$.
    Infer that sampling each edge of $G$ with probability $p := L/\lambda$ gives us a connected graph with high probability.
\end{alphenum}

\subsection{Solution}
\begin{proof}

Consider a color set $C$ with cardinality $|C| := Q$. Let us color each vertex of $G = (V,E)$ uniformly at random:

$$\chi(v) \stackrel{\text{i.i.d.}}{\sim} \mathcal{U}\{Q\}, \quad \forall v \in V$$

For each vertex $v \in V$ all colours $c \in C$, and any subset $S \subseteq N(v)$ of size $|S| = \beta + 1$, define the bad event:

$$B(v, c, S) = \{\text{all vertices in S have color } c\}$$

If such a bad event exists, then by definition, the coloring $\chi$ is not $\beta$-frugal over $G$.

From the uniform color assignment, we have that the probability $p$ of the bad event occuring for a fixed tuple $(v,c,S)$ is given by:

$$\mathbb{P}[B(v,c,S)] = Q^{-(\beta + 1)} \stackrel{\triangle}{=} p$$

Let us now find an upper-bound for the number of events $d^\star$ on which $B(u,c,S)$ depends, for a fixed vertex $u \in S$:
\begin{enumerate}
    \item Because each color is assigned independently for each vertex, the event $B(u,c,S)$ only depends on the colors in $S$. In other words, for the events $B(u,c,S)$ and $B(w,\tilde c, \tilde S)$ to be dependent, it must be the case that $u \in \tilde S \subseteq N(w)$, i.e. $w$ and $u$ must be neighbors.
    
    By hypothesis, we know that $\delta(v) \le \Delta$ for all $v \in V$, so there are at most $\Delta$ choices for an initial vertex. 

    \item We have at most $\left( \begin{smallmatrix} \Delta - 1 \\ \beta \end{smallmatrix}\right)$ choices for the remaining $\beta$ vertices in $\tilde S \subseteq N(w) \backslash\{u\}$.

    \item We have $Q$ choices for the color $\tilde c \in C$.
\end{enumerate}
Thus, 

$$d^\star \le \Delta 
\begin{pmatrix}
    \Delta -1 \\
    \beta
\end{pmatrix} 
Q -1$$

Where we subtracted $1$ for the event itself. Union-bounding over the fixed cardinality $|S| = \beta + 1$ yields an upper-bound to the total number of events $d$ that any non-fixed event depends on:

$$
\begin{align*}
d &\le (\beta + 1)d^\star \\
&\le Q(\beta +1)\Delta \begin{pmatrix} \Delta -1 \\ \beta\end{pmatrix} -1 \\
&\le Q (\beta +1) \Delta \left( \frac{e\Delta}{\beta}\right)^\beta -1\\
&= Q(\beta +1)\left(\frac{e}{\beta}\right)^\beta \Delta^{\beta + 1} -1 \stackrel{\triangle}{=} QC_\beta \Delta^{\beta + 1} -1
\end{align*}
$$

Where $C_\beta = (\beta +1)\left(\frac{e}{\beta}\right)^\beta$, and where we used stirling's bound in the third line:

$$k! \ge \left( \frac{k}{e}\right)^k \Longrightarrow \frac{n^k}{k!} \le \frac{n^k}{(k/e)^k} = \left( \frac{en}{k}\right)^k$$

The Local Lovász Lemma states that, if $ep(d+1) \le 1$, then with positive probability, no bad event occurs, meaning a $\beta$-frugal coloring exists. 
Thus, plugging in the computed value for $p$ and the above bound for $d$ into the LLL inequality, it suffices to have

$$
eQ^{-(\beta +1)} \cdot QC_\beta \Delta^{\beta + 1} \le 1 \
\Longrightarrow eC_\beta Q^{-\beta}\Delta^{\beta +1} \le 1 \\
$$

Finally, solving for $Q$ gives:

$$Q^\beta \ge eC_\beta \Delta^{\beta +1} \Longrightarrow Q \ge (eC_\beta)^{1/\beta } \Delta^{1 + 1/\beta}$$

And thus
$$Q =O\left( \Delta^{1 + 1/\beta}\right)$$
as required. 


\end{proof}

\section{Problem 2 (Nearly Orthonormal Vectors)}
Call a set of unit vectors
``near-orthonormal'' if the inner product of any two of them is close
to zero. In this problem we will show that while there are at most $d$
orthonormal vectors in $\R^d$, there can be \underline{exponentially} many
near-orthonormal vectors! For vectors $x,y \in \R^d$, we use $\langle
    x,y \rangle = \sum_{i = 1}^d x_iy_i$ to denote the inner product.

\begin{alphenum}
    %   \item[(a)] Use the Chernoff-Hoeffding bound for $[0,1]$-r.v.s from
    %     Lecture~\#18 to prove the following concentration bound for $\{+1,
    %     -1\}$-r.v.s.  Let $Y_1, Y_2, \ldots, Y_n$ be independent and
    %     identical $\{-1,+1\}$-valued random variables, each $Y_i = 1$ with
    %     probability $1/2$ and $Y_i = -1$ w.p. $1/2$. Let $Y = \sum_{i = 1}^n
    %     Y_i$. Prove that for $\lambda \leq n$,
    %     \[ \Pr[ |Y| \geq \lambda ] \leq 2\exp\left( - \frac{ \lambda^2 }{ 6n
    %       } \right) \]

    %     \answer{Define $X_i = \frac{1 + Y_i}{2}$, and $X = \sum_i X_i$. Then
    %       $Y_i = 2X_i - 1$.  Clearly, these $X_i$'s are $n$ independent
    %       $\{0,1\}$-valued r.v.s with $E[X_i] = 1/2$, and
    %       \[ \Pr[ |\sum_i Y_i| \geq \lambda ] = \Pr[ |\sum_i X_i - n/2 |
    %       \geq \lambda/2 ]. \] Now we can apply Hoeffding's bound to get
    %       that the probability
    %       \begin{align*}
    %         \Pr[ |\sum_i Y_i| \geq \lambda ] &= 
    %         \Pr[ \sum_i X_i - n/2 \geq \lambda/2 ] +
    %         \Pr[ \sum_i X_i - n/2 \leq -\lambda/2 ] \\
    %         & \leq \exp\left( - \frac{(\lambda/2)^2}{2(n/2) + \lambda}\right)
    %         + \exp \left( - \frac{(\lambda/2)^2}{3(n/2)}\right) \\
    %         & \leq 2 \exp( - \lambda^2/(6n)).
    %       \end{align*}
    %     }

    \item Let $x = (x_1, x_2, \ldots, x_d)$ and $y = (y_1, y_2,
        \ldots, y_d)$ be two independently and uniformly chosen vectors in
    $\{-1,1\}^d$.  (I.e., each bit $x_i$ and $y_i$ in each vector is
    independently and uniformly chosen from $\{-1,1\}$.) Show that
    \[ \Pr[ | \langle x,y \rangle | \geq \varepsilon d ] \leq 2 \exp \left( -
        \varepsilon^2 d/6 \right) \]
    % (Hint: derive a concentration bound for $\{-1, +1\}$-valued random
    % variables from the main concentration bound in Lecture~\#18.)

    \item Given parameter $\varepsilon > 0$, a set $S$ of unit vectors is
    called \emph{$\varepsilon$-orthonormal} if for all $\vec{x}, \vec{y} \in S$,
    \[ |\langle \vec{x}, \vec{y} \rangle| \leq \varepsilon . \] %  $\|
    %     \vec{u} \|$ is the Euclidean norm.

    %     Here's one way to construct such a set $S$.
    Show that there exists a constant $c > 0$ and constant $d_0$, such that for any
    $\varepsilon \leq 1/2$ (say) and any $d \geq d_0$, if you sample
    $N := \exp(c\varepsilon^2 d)$ random vectors independently
    and uniformly from the set
    $\{-\frac{1}{\sqrt{d}},+\frac{1}{\sqrt{d}}\}^d$, this sampled set
    is $\varepsilon$-orthonormal with probability at least $1/2$.

    % \item Let $S$ be any set of \emph{$\varepsilon$-orthonormal}
    %   vectors. For any $x \in \R^d$, define the ball
    %   \[ B(x,r) := \{y \mid \| x - y \|_2 \leq r\}. \]
    %   \begin{enumerate}
    %   \item Prove: the Euclidean distance between any two points in $S$
    %     lies in $[\sqrt{2(1-\e)}, \sqrt{2(1+\e)}]$. Hence for any point
    %     $x_0 \in S$, the entire set lies in $S \sse B(x_0, \sqrt{2(1+\e)})$.
    %   \item For any $x \neq y \in S$, show that $B(x, \sqrt{(1-\e)/2})
    %     \cap B(y, \sqrt{(1-\e)/2}) = \emptyset$.
    %   \item
    %   \end{enumerate}

\end{alphenum}


\subsection{Solution}
We aim to prove that $L$ is a Lipschitz function, and then apply McDiarmid's inequality to obtain the desired concentration bound.

\begin{proof}
We claim that $L$ is a Lipschitz function.
That is, for any two configurations $X=(X_1, \dots, X_n)$ and $X'=(X_1, \dots, X_i', \dots, X_n)$ that differ only in one coordinate $i$, the following holds:
$$
|L(X) - L(X')| \le c
$$
for some constant $c$.

Let $T$ be the minimum spanning tree (MST) of $X$, and let $T'$ be the tree obtained from $T$ by replacing the vertex $X_i$ with $X_i'$.
Although $T'$ may not be the MST of $X'$, it is still a valid spanning tree on the modified set of points.
let $T'^\star$ denote the MST of $X'$, then:
$$
\operatorname{len}(T'^\star) \le \operatorname{len}(T')
$$

Since all points lie within the unit square $[0,1]^2$, the maximum Euclidean distance between any two point is
$$
\sqrt{1^2+1^2}=\sqrt{2}
$$

Given the fact that in planar MST, each vertex has maximum degree of 5.
Therefore, when a single point $X_i$ is moved to $X_i'$, at most five edges are affected, each by changing by at most $\sqrt{2}$ in length.
Hence
$$
|\operatorname{len}(T) - \operatorname{len}(T')| \le 5\sqrt{2}
$$
Since $\operatorname{len}(T'^\star) \le \operatorname{len}(T')$, it follows that:
$$
|L(X) - L(X')| \le |\operatorname{len}(T) - \operatorname{len}(T'^\star)| \le 5\sqrt{2}
$$
Thus, $L$ is $c$-Lipschitz with $c=5\sqrt{2}$.

We can now apply McDiarmid's inequality, which states that if $L$ is $c_i$-Lipschitz in each coordinate, then:
$$
\Pr\big[ |L - \mu| \geq t \big] \leq 2\exp\!\left( \frac{-2t^2}{\sum_{i=1}^n c_i^2} \right)
$$
Since each $c_i=5\sqrt{2}$, we have
$$
\sum_{i=1}^n c_i^2 = n\cdot(5\sqrt{2})^2 = 50n
$$
Setting $t=\varepsilon n$, we obtain:
$$
\Pr\big[ |L - \mu| \geq \varepsilon n \big] \leq 2\exp\!\left( \frac{-2\varepsilon^2n^2}{50n} \right) = 2\exp\!\left( \frac{-\varepsilon^2n}{25} \right)
$$

\end{proof}

\section{Problem 3 (Packet Scheduling with Randomized Delays)}


We consider the problem of routing $k$ packets in a network, where
packet $i$ goes from $s_i$ to $t_i$, along a fixed $s_i$-$t_i$ path
$P_i$. (These paths may be chosen, e.g., using randomized rounding as seen in class, or
some other approach---but this does not matter for us.)
% seen in class).
We assume a synchronous model where time proceeds in discrete steps
($t=0, 1, 2, \dots$). For feasibility, each edge should be used by at
most one packet per timestep. Define two key parameters based on the given paths:
\begin{itemize}
    \item \textbf{Congestion ($C$)}: The maximum number of paths $P_i$
          that use any single edge.
    \item \textbf{Dilation ($D$)}: The maximum length (number of
          edges) of any path $P_i$.
\end{itemize}

In the \emph{na\"{\i}ve schedule}, packet $i$ tries to cross the
$j^{th}$ edge of $P_i$ at time $j$. This takes time at most $D$ to
complete, but could be infeasible: as many as $C$ packets may try to use an
edge at the same time. Let us see how to get schedules which take slightly
longer to complete, but have less congestion.


Consider the following randomized approach for defining a
\emph{nominal schedule}.
% This schedule ignores the real-world constraint that only one packet can cross an edge at a time (i.e., it assumes infinite edge capacity), but our goal is to show that this nominal schedule has low congestion with high probability.
\begin{enumerate}
    \item[] \textbf{Random Delay:} Each packet $i$ independently chooses an initial integer delay $\Delta_i$ uniformly at random from the set $\{0, 1, 2, \dots, C\}$. (Note: There are $C+1$ choices.)
    \item[] \textbf{Nominal Schedule:} Suppose path $P_i=(e_{i,1}, e_{i,2}, \dots)$. In the nominal schedule, packet $i$ crosses edge $e_{i,j}$ at time $\Delta_i + j$.
\end{enumerate}
%\vspace{0.1in}
I.e., each packet waits for a random amount $\Delta_i$, and then moves
one edge per timestep after that. Of course, this may still cause
packets to use an edge at the same time. Define the \emph{nominal
    congestion} $X_{e,t}$ as the number of packets whose nominal
schedule requires them to cross edge $e$ at time $t$. We want to show
that the maximum nominal congestion is small:
\begin{alphenum}
    \item Argue that $X_{e,t}$ is a sum of independent random variables,
    and use a Chernoff bound plus union bound to show that with high
    probability, the maximum nominal congestion
    $R_{\max} = \max_{e,t} X_{e,t}$ is $O(\log(k\cdot D\cdot C))$.
    %   \item Use the Lov\'asz Local Lemma to find a schedule where the
    %         maximum nominal congestion $R_{\max} = \max_{e,t} X_{e,t}$ is
    %         $O(\log(D\cdot C))$; note that we do not have a dependence on $k$
    %         any more.
    %   \item We have taken the trivial schedule where \alert{Complete this}
\end{alphenum}


\bigskip
\emph{Remark:} Starting with the nominal schedule, greedily scheduling the packets (where packets wait if an edge is busy) can be shown to result in a schedule length bounded using this result for $R_{\max}$, specifically $O(C + D \cdot R_{\max})$. This result is relatively close to the lower bound $\max(C, D)$. A landmark result by Leighton, Maggs, and Rao (1988) used a much more powerful probabilistic tool, the Lovász Local Lemma (which we will cover in lecture \#5), to prove that there exists a choice of delays such that the schedule length is $O(C+D)$. This is a constant factor approximation of the optimal schedule.






\subsection{Solution}
\subsubsection{Part A}
\begin{proof}
    
Pick an arbitrary assignment $x_t$ with distance $D_t > 0$ from a satisfying $x^\star$, and assume $x_t$ does not satisfy $\Phi$ (in which case the algorithm would terminate).

Because $x^\star$ satisfies $\Phi$, every clause $C$ contains at least one literal $\ell^\star \in C$ set to true under the assignment $x^\star$. Conversely, as $C$ is unsatisfied under $x_t$, every literal of $C$ must be false in $x_t$. In particular, we have that $\ell^\star \in C$ is false in $x_t$. Because the algorithm picks a literal $\ell \in C$ u.a.r. among the $k$ possible choices, and at least one of them ($\ell^\star$) would decrease the Hamming distance by $1$, it follows that 
$$\Pr[D_{t+1} = D_t - 1 \,|\, x_t] \ge \frac{1}{k}$$
\end{proof}


\subsubsection{Part B}

\begin{proof}
We assume the satisfying assignment $x^\star$ is unique, as per clarifications on the exercise. 

Consider the event that each of the first $d$ steps decrease the distance by exactly $1$. From part $(a)$, we have that 

$$\Pr\left[ \bigcap_{t=0}^{d-1} \{D_{t+1} = D_t -1\}\right] = \prod_{t=0}^{d-1} \Pr[D_{t+1} = D_t - 1 \,|\, x_t] \ge \left( \frac{1}{k}\right)^d$$

Thus:

$$\Pr[\text{ hit } 0 \text{ within } d \text{ steps} \, | \, D_0 = d \,] \ge \left( \frac{1}{k}\right)^d$$
\end{proof}


\subsubsection{Part C}

\begin{proof}
the result from part $(b)$ gives us

$$\Pr[\text{ success in one try } \, | \, D_0 = d \,] \ge \left( \frac{1}{k}\right)^d$$

From which it immediately follows that 

$$
\begin{align*}
\Pr[ \text{ success in one try }] \ge \mathbb{E}\left[ \left(\frac{1}{k}\right)^{D_0}\right] &= \prod_{i=1}^n \mathbb{E}\left[ \left(\frac{1}{k}\right)^{Z}\right], \quad Z \sim \text{Ber}(1/2) \\
&= \prod_{i=1}^n \left( \frac{1}{2} + \frac{1}{2} \cdot \frac{1}{k}\right) \\
&= \left( \frac{k + 1}{2k}\right)^n
\end{align*}
$$
Where we simply used the MGF of $D_0 \sim \text{Bin}(n, 1/2).$
\end{proof}


\subsubsection{Part D}

The event from $(b)$, namely:
$$D_0 = d \Longrightarrow \Pr[\text{ hit 0 within } d \text{ steps }] \ge (1/k)^d$$
is fully contained in the first $d$ steps. Because we consider the K-SAT problem over $n$ variables, it is clear that the maximal possible initial distance is $D_0 = n$, i.e. when all variables from $x_0 \sim \mathcal{U}\{0,1\}^n$ differs from the satisfying assignment $x^\star$. Thus, $D_0 = d \le n$, and choosing $T=n$ suffices to capture the event fully. 

From part $(c)$, we know that the probability of success in one try is:

$$P \ge \left( \frac{k+1}{2k}\right)^n$$

Therefore the expected number of necessary tries until success is 

$$\mathbb{E}[\text{necessary tries until success}] = O(1/P) = O\left( \left(\frac{2k}{k+1}\right)^n\right)$$

Each try costs $T=n$ time, and thus:

$$\mathbb{E}[\text{time}] = O\left(n \left(\frac{2k}{k+1}\right)^n\right) = \tilde O \left( \left(\frac{2k}{k+1}\right)^n\right)$$






\end{document}
